\subsection{Metodos numericos}
    \subsubsection{FFT}
    \pathinputminted[tabsize=2,breaklines,firstline=5,lastline=84,fontsize=\small]{c++}{numeric_methods.cpp}
        
        % \subsubsection{FFT con raíces de la unidad complejas}
        % % \inputminted[tabsize=2,breaklines,firstline=13,lastline=44,fontsize=\small]{c++}{fft.cpp}
        
        % \subsubsection{FFT con raíces de la unidad discretas (NTT)}
        % % \inputminted[tabsize=2,breaklines,firstline=46,lastline=90,fontsize=\small]{c++}{fft.cpp}
        % 	\subsubsection{Otros valores para escoger la raíz y el módulo}
        % 		\begin{table}[H]
        % 			\centering
        % 			\begin{tabular}{|p{2cm}|p{1.7cm}|p{2cm}|p{4.5cm}|}
        % 				\hline
        % 				Raíz $n$-ésima de la unidad ($\omega$) & $\omega^{-1}$ & Tamaño máximo del arreglo ($n$) & Módulo $p$ \\ \hline
        % 				15 & 30584 & $2^{14}$ & $4 \times 2^{14} + 1 = 65537$ \\ \hline
        % 				9 & 7282 & $2^{15}$ & $2 \times 2^{15} + 1 = 65537$ \\ \hline
        % 				3 & 21846 & $2^{16}$ & $1 \times 2^{16} + 1 = 65537$ \\ \hline
        % 				8 & 688129 & $2^{17}$ & $6 \times 2^{17} + 1 = 786433$ \\ \hline
        % 				5 & 471860 & $2^{18}$ & $3 \times 2^{18} + 1 = 786433$ \\ \hline
        % 				12 & 3364182 & $2^{19}$ & $11 \times 2^{19} + 1 = 5767169$ \\ \hline
        % 				\textbf{5} & \textbf{4404020} & $\mathbf{2^{20}}$ & $7 \times 2^{20} + 1 = \textbf{7340033}$ \\ \hline
        % 				38 & 21247462 & $2^{21}$ & $11 \times 2^{21} + 1 = 23068673$ \\ \hline
        % 				21 & 49932191 & $2^{22}$ & $25 \times 2^{22} + 1 = 104857601$ \\ \hline
        % 				4 & 125829121 & $2^{23}$ & $20 \times 2^{23} + 1 = 167772161$ \\ \hline
        % 				\textbf{31} & \textbf{128805723} & $\mathbf{2^{23}}$ & $119 \times 2^{23} + 1 = \textbf{998244353}$ \\ \hline
        % 				2 & 83886081 & $2^{24}$ & $10 \times 2^{24} + 1 = 167772161$ \\ \hline
        % 				17 & 29606852 & $2^{25}$ & $5 \times 2^{25} + 1 = 167772161$ \\ \hline
        % 				30 & 15658735 & $2^{26}$ & $7 \times 2^{26} + 1 = 469762049$ \\ \hline
        % 				137 & 749463956 & $2^{27}$ & $15 \times 2^{27} + 1 = 2013265921$ \\ \hline
        % 			\end{tabular}
        % 		\end{table}
        
        
    % \subsubsection{Aplicaciones}
    % 	\subsubsection{Multiplicación de polinomios}
    % 	\pathinputminted[tabsize=2,breaklines,firstline=72,lastline=84,fontsize=\small]{c++}{numeric_methods.cpp}
        
    % 	\subsubsection{Multiplicación de números enteros grandes}
    % 	% \inputminted[tabsize=2,breaklines,firstline=120,lastline=156,fontsize=\small]{c++}{fft.cpp}
        
    % 	\subsubsection{Inverso de un polinomio}
    % 	% \inputminted[tabsize=2,breaklines,firstline=158,lastline=184,fontsize=\small]{c++}{fft.cpp}
        
    % 	\subsubsection{Raíz cuadrada de un polinomio}
    % 	% \inputminted[tabsize=2,breaklines,firstline=186,lastline=208,fontsize=\small]{c++}{fft.cpp}
    
    
        % \subsubsection{A Bitwise Convolution}
        % \subsubsection{Möbius inversion}
        % \subsubsection{Dirichlet convolution}
        % \subsubsection{Schonhage-Strassen}
        % \subsubsection{Integration by Simpson's formula}
        % \subsubsection{Newton's method for finding roots}
        % \subsubsection{Ternary Search}
