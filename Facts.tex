
\section{Facts} 
\subsection{Números de Catal\'an} están definidos por la recurrencia:
\begin{equation*}
  C_{n+1} = \sum_{i=0}^nC_iC_{n-i}
\end{equation*}

%Una fórmula cerrada para los números de Catalán es:
\begin{equation*}
  C_n = \frac{1}{n+1}\binom{2n}{n} = \binom{2n}{n} - \binom{2n}{n+1}
\end{equation*}

\subsection{Números de Stirling de primera clase} son el número de permutaciones de $n$ elementos con exactamente $k$ ciclos disjuntos.
\begin{equation*}
  \stirlingfirst{n}{k} = (n-1)\stirlingfirst{n-1}{k} + \stirlingfirst{n-1}{k-1}
\end{equation*}

\subsection{Números de Stirling de segunda clase} son el número de particionar un conjunto de
$n$ elementos en $k$ subconjuntos no vacíos.
\begin{equation*}
  \stirlingsecond{n}{k} >= k\stirlingsecond{n-1}{k} + \stirlingsecond{n-1}{k-1}
\end{equation*}

Además:
\begin{equation*}
  \stirlingsecond{n}{k} = \frac{1}{k!}\sum_{j=0}^k (-1)^{k-j} \binom{k}{j} j^n
\end{equation*}
\newpage
\subsection{Números de Bell} cuentan el número de formas de dividir $n$ elementos en subconjuntos.

\begin{equation*}
  \mathcal{B}_{n+1} = \sum_{k=0}^n \binom{n}{k} \mathcal{B}_k
\end{equation*}

\

\begin{tabular}{|c|c|c|c|c|c|c|c|c|c|c|c|}
  \hline
  x&0&1&2&3&4&5&6&7&8&9&10 \\ \hline %&9&10&11&12
  $\mathcal{B}_x$&1&1&2&5&15&52&203&877&4.140&21.147&115.975 \\ \hline %&678.570&4.213.597
\end{tabular}

\subsection{Derangement} permutación que no deja ningún elemento en su lugar original

\begin{equation*}
  !n = (n - 1)( !(n - 1) + !(n - 2) ); !1 = 0, !2 = 1
\end{equation*}

\begin{equation*}
  !n = n! \sum_{k = 0}^n \frac{(-1)^k}{k!}
\end{equation*}

\subsection{Números armónicos}

\begin{equation*}
H_n = \sum_{k = 1}^n \frac{1}{k}
\end{equation*}

\begin{equation*}
\frac{1}{2n+1} < H_n - \ln n - \gamma < \frac{1}{2n}
\end{equation*}

\begin{equation*}
\gamma = 0.57721 56649 01532 86060 65120 90082 40243 10421 59335 \ldots
\end{equation*}
\newpage
\subsection{Número de Fibonacci} $f_0 = 0$, $f_1 = 1$:

\begin{equation*}
f_n = \frac{1}{\sqrt{5}}(\frac{1 + \sqrt{5}}{2})^n - \frac{1}{\sqrt{5}}(\frac{1 - \sqrt{5}}{2})^n
\end{equation*}

\begin{equation*}
  f_{n+1} = f_n * 2 - f_{n-2}
\end{equation*}

\begin{equation*}
  f_0 + f_1 + f_2 + \dots + f_n = f_{n+2} -1
\end{equation*}

\begin{equation*}
  f_0 - f_1 + f_2 - \dots + (-1)^n f_n = (-1)^n f_{n-1} - 1
\end{equation*}

\begin{equation*}
  f_1 + f_3 + f_5 + \dots + f_{2n-1} = f_{2n}
\end{equation*}

\begin{equation*}
  f_0 + f_2 + f_4 + \dots + f_{2n} = f_{2n+1} - 1
\end{equation*}

\begin{equation*}
  f_0^2 + f_1^2 + f_2^2 + \dots + f_n^2 = f_n f_{n+1}
\end{equation*}

\begin{equation*}
  f_1 f_2 + f_2 f_3 + f_3 f_4 + \dots + f_{2n-1} f_n = f_{2n}^2
\end{equation*}

\begin{equation*}
  f_1 f_2 + f_2 f_3 + f_3 f_4 + \dots + f_{2n} f_{2n+1} = f_{2n+1}^2 - 1
\end{equation*}

\begin{equation*}
  k \ge 1 \Rightarrow f_{n+k} = f_k f_{n+1} + f_{k-1} f_n    \forall n \ge 0
\end{equation*}\\

Identidad de Cassini: $f_{n+1} f{n-1} - f_n^2 = (-1)^n$



\begin{equation*}
  f_{n+1}^2 + f_n^2 = f_{2n + 1}	
\end{equation*}

\begin{equation*}
  f_{n+2}^2 - f_n^2 = f_{2n + 2}
\end{equation*}

\begin{equation*}
  f_{n+2}^2 - f_{n+1}^2 = f_n f_{n + 3}
\end{equation*}

\begin{equation*}
  f_{n+2}^3 - f_{n+1}^3 - f_n^3 = f_{3n + 3}
\end{equation*}

\begin{equation*}
  mcd(f_n, f_m) = f_{mcd(n, m)}
\end{equation*}

\begin{equation*}
   f_{n+1} = \sum_{j = 0}^{\lfloor \frac{n}{2} \rfloor} \binom{n-j}{j}
\end{equation*}

\begin{equation*}
f_{3n} = \sum_{j = 0}^{n} \binom{n}{j} 2^j f_j
\end{equation*}

El último dígito de cada número se repite periódicamente cada $60$ números. Los dos últimos, cada $300$; a partir de ahí, se repiten cada $15*10^{n-1}$ números.

\subsection{Sumas de combinatorios}

\begin{equation*}
\sum_{i = n}^m \binom{i}{n} = \binom{m + 1}{n + 1}
\end{equation*}

\begin{equation*}
\sum_{i = 0}^k \binom{n}{i} \binom{m}{k - i} = \binom{n + m}{k}
\end{equation*}


\subsection{Funciones generatrices}
Una lista de funciones generatrices para secuencias útiles:

\ 

\begin{tabular}{|c|c|}
  \hline
  $(1,1,1,1,1,1,\ldots)$ & $\frac{1}{1-z}$ \\ \hline
  $(1,-1,1,-1,1,-1,\ldots)$ & $\frac{1}{1+z}$ \\ \hline
  $(1,0,1,0,1,0,\ldots)$ & $\frac{1}{1-z^2}$ \\ \hline
  $(1,0,\ldots,0,1,0,1,0,\ldots,0,1,0,\ldots)$ & $\frac{1}{1-z^2}$ \\ \hline
  $(1,2,3,4,5,6,\ldots)$ & $\frac{1}{(1-z)^2}$ \\ \hline
  $(1,\binom{m+1}{m},\binom{m+2}{m},\binom{m+3}{m},\ldots)$ & $\frac{1}{(1-z)^{m+1}}$ \\ \hline
  $(1,c,\binom{c+1}{2},\binom{c+2}{3},\ldots)$ & $\frac{1}{(1-z)^c}$ \\ \hline
  $(1,c,c^2, c^3, \ldots)$ & $\frac{1}{1-cz}$ \\ \hline
  $(0,1,\frac{1}{2},\frac{1}{3},\frac{1}{4},\ldots)$ & $\ln \frac{1}{1-z}$ \\ \hline
\end{tabular}

\ 

Truco de manipulación:
\begin{equation*}
  \frac{1}{1-z}G(z) = \sum_{n}\sum_{k\leq n}g_kz^n
\end{equation*}

\subsection{The twelvefold way} ¿Cuántas funciones $f \colon N \rightarrow X$ hay?

\ 

\begin{tabular}{|c|c|c|c|c|}
  \hline
  $N$ & $X$ & Any $f$ & Injective & Surjective \\ \hline
  dist. & dist. & $x^n$ & $(x)_n$ & $x! \stirlingsecond{n}{x}$ \\ \hline
  indist. & dist. & $\binom{x+n-1}{n}$ & $\binom{x}{n}$ & $\binom{n-1}{n-x}$ \\ \hline
  dist. & indist. & $\stirlingsecond{n}{1} + \ldots + \stirlingsecond{n}{x}$ & $[n \leq x]$ & $\stirlingsecond{n}{k}$ \\ \hline
  indist. & indist. & $p_1(n) + \ldots p_x(n)$ & $[n \leq x]$ & $p_x(n)$ \\ \hline
\end{tabular}

\ 

Where $\binom{a}{b} = \frac{1}{b!}(a)_b $ and $p_x(n)$ is the number of ways to partition the integer $n$ using $x$ summands.

\subsection{Teorema de Euler} si un grafo conexo, plano es dibujado sobre un plano sin intersección de aristas,
y siendo v el número de vértices, e el de aristas y f la cantidad de caras (regiones conectadas por aristas,
incluyendo la región externa e infinita), entonces
\begin{equation*}
v-e+f = 2
\end{equation*}

\subsection{Burnside's Lemma} Si X es un conjunto finito y G es un grupo de permutaciones que actúa sobre X, sean
$S_x = \{g \in G:g*x=x\}$ y $Fix(g) = \{x \in X:g*x=x\}$. Entonces el número de órbitas está dado por:
\begin{equation*}
N = \frac{1}{|G|}\sum_{x \in X}|S_x| = \frac{1}{|G|}\sum_{g \in G}|Fix(g)|
\end{equation*}

\subsection{Ángulo entre dos vectores} Sea $\alpha$ el ángulo entre $\vec{a}$ y $\vec{b}$:
\begin{equation*}
\cos \alpha = \frac{\vec{a} \cdot \vec{b}}{\norm{\vec{a}} \norm{\vec{b}}}
\end{equation*}

\subsection{Proyección de un vector} Proyección de $\vec{a}$ sobre $\vec{b}$:
\begin{equation*}
\text{proy} _ {\vec{b}} \vec{a} = (\frac{\vec{a} \cdot \vec{b}}{\vec{b} \cdot \vec{b}}) \vec{b}
\end{equation*}