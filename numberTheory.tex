
\section{Number Theory}
\enlargethispage*{\baselineskip}

  \pathinputminted{latex}{aa.tex}
  \subsection{Aritmetica modular}	
			\subsubsection{Inverso modular}
			\pathinputminted[tabsize=2,breaklines,firstline=84,lastline=88,fontsize=\small]{c++}{Number_theory.cpp}						
			\pathinputminted[tabsize=2,breaklines,firstline=92,lastline=100,fontsize=\small]{c++}{Number_theory.cpp}						
			
			\subsubsection{Linear Congruence Equation}
			\pathinputminted[tabsize=2,breaklines,firstline=112,lastline=122,fontsize=\small]{c++}{Number_theory.cpp}						
			
			\subsubsection{Factorial modulo p}
			\pathinputminted[tabsize=2,breaklines,firstline=161,lastline=170  ,fontsize=\small]{c++}{Number_theory.cpp}						

			\subsubsection{Chinese Remainder Theorem}
			\pathinputminted[tabsize=2,breaklines,firstline=177,lastline=192,fontsize=\small]{c++}{Number_theory.cpp}

% 			\subsubsection{Discrete Root}
% 			The problem of finding a discrete root is defined as follows. 
% 			Given a prime n and two integers a and k, find all x for which:\\
% 			$$x^{k} \enskip (mod \enskip m)$$ 
		
% \subsubsection{Primitive Root}

% 			\subsubsection{Discrete Logarithm}

% 			\subsubsection{Montgomery Multiplication}

	\subsection{Cribas y primos}
			\subsubsection{Criba de eratostenes}
			\pathinputminted[tabsize=2,breaklines,firstline=26,lastline=36,fontsize=\small]{c++}{Primes.cpp}
			
			\subsubsection{Criba de factor primo más pequeño}
			\pathinputminted[tabsize=2,breaklines,firstline=55,lastline=66,fontsize=\small]{c++}{Primes.cpp}
						
			\subsubsection{Criba de la función $\varphi$ de Euler}
			\pathinputminted[tabsize=2,breaklines,firstline=79,lastline=88,fontsize=\small]{c++}{Primes.cpp}
			
			\subsubsection{Criba de la función $\mu$}
			\pathinputminted[tabsize=2,breaklines,firstline=90,lastline=98,fontsize=\small]{c++}{Primes.cpp}
			
			\subsubsection{Triángulo de Pascal}
			\pathinputminted[tabsize=2,breaklines,firstline=100,lastline=110,fontsize=\small]{c++}{Primes.cpp}
			
			% \subsubsection{Segmented sieve}
			% \inputminted[tabsize=2,breaklines,firstline=860,lastline=892,fontsize=\small]{c++}{numberTheory.cpp}
			
			\subsubsection{Criba de primos lineal}
			\pathinputminted[tabsize=2,breaklines,firstline=41,lastline=53,fontsize=\small]{c++}{Primes.cpp}
			
			% \  subsubsection{Criba lineal para funciones multiplicativas}
			% \inputminted[tabsize=2,breaklines,firstline=827,lastline=858,fontsize=\small]{c++}{numberTheory.cpp}
			
			\subsubsection{Block sieve}
			\pathinputminted[tabsize=2,breaklines,firstline=126,lastline=157,fontsize=\small]{c++}{Primes.cpp}

			\subsubsection{Prime factors of $n!$}
			if $p$ is prime the highest power $p^{k}$ of $p$ that divides $n!$ is given by 
			\begin{align*}
				k = \floor*{\frac{n}{p}}  + \floor*{\frac{n}{p^{2}}} +
				 \floor*{\frac{n}{p^{3}}} + \cdots	
			\end{align*}


			\subsubsection{Primaly test(miller rabin)}
			\pathinputminted[tabsize=2,breaklines,firstline=184,lastline=232,fontsize=\small]{c++}{Primes.cpp}
			
			% \subsubsection{Potencia de un primo que divide a un factorial}
			% \pathinputminted[tabsize=2,breaklines,firstline=99,lastline=109,fontsize=\small]{c++}{Primes.cpp}

			% \subsubsection{Factorización de un número}
			% \pathinputminted[tabsize=2,breaklines,firstline=99,lastline=109,fontsize=\small]{c++}{Primes.cpp}

			% \subsubsection{Factorización de un factorial}
			% \pathinputminted[tabsize=2,breaklines,firstline=99,lastline=109,fontsize=\small]{c++}{Primes.cpp}

			\subsubsection{Factorización varios metodos}
			\pathinputminted[tabsize=2,breaklines,firstline=302,lastline=485,fontsize=\small]{c++}{Primes.cpp}

			\subsubsection{Factorizacion usando todos los metodos}
			\pathinputminted[tabsize=2,breaklines,firstline=486,lastline=514,fontsize=\small]{c++}{Primes.cpp}

			\subsubsection{Numero de divisores hasta $10^{18}$}
			\pathinputminted[tabsize=2,breaklines,firstline=268,lastline=299,fontsize=\small]{c++}{Primes.cpp}
		
			\subsection{Funciones multiplicativas}
			\subsubsection{Función $\varphi$ de Euler}
			\pathinputminted[tabsize=2,breaklines,firstline=123,lastline=135,fontsize=\small]{c++}{Number_theory.cpp}
			The most famous and important property of Euler's totient function
			 is expressed in \textbf{Euler's theorem:}
			\\
			\begin{align}
				\alpha^{\phi(m)} \equiv 1 (mod  \quad m)
			\end{align}
			if \textbf{\textit{$\alpha$}}
			and \textbf{\textit{m}} 
			are relative prime.\\
			In the particular case when m is prime,
			Euler's theorem turns into \textbf{Fermat's little theorem:}
			\begin{align}
				\alpha^{m-1}\equiv 1 (mod  \quad m)
			\end{align}
			\smallskip
			\begin{align}
				\alpha^{n}\equiv \alpha^{n \enskip mod \enskip\phi(m)} \quad (mod  \quad m)
			\end{align}
			This allows computing $x^{n} mod \quad m$ for very big $n$, especially if
			n is the result of another computation, 
			as it allows to compute n under a modulo.
			% \newpage
			% \subsubsection{Función $\varphi^-1$ de Euler}
				% \subsubsection{Función $\sigma$}
				% % \inputminted[tabsize=2,breaklines,firstline=209,lastline=226,fontsize=\small]{c++}{numberTheory.cpp}
				
				% \subsubsection{Función $\Omega$}
				% % \inputminted[tabsize=2,breaklines,firstline=228,lastline=235,fontsize=\small]{c++}{numberTheory.cpp}
				
				% \subsubsection{Función $\omega$}
				% % \inputminted[tabsize=2,breaklines,firstline=237,lastline=244,fontsize=\small]{c++}{numberTheory.cpp}
								
				% \subsubsection{Función $\mu$}
				% % \inputminted[tabsize=2,breaklines,firstline=276,lastline=287,fontsize=\small]{c++}{numberTheory.cpp}
				
				% \subsubsection{Number of divisors / sum of divisors}
			

		% \subsection{Orden multiplicativo, raíces primitivas y raíces de la unidad}
		% 	\subsubsection{Función $\lambda$ de Carmichael}
		% 	% \inputminted[tabsize=2,breaklines,firstline=260,lastline=274,fontsize=\small]{c++}{numberTheory.cpp}
			
		% 	\subsubsection{Orden multiplicativo módulo $m$}
		% 	% \inputminted[tabsize=2,breaklines,firstline=289,lastline=305,fontsize=\small]{c++}{numberTheory.cpp}
			
		% 	\subsubsection{Número de raíces primitivas (generadores) módulo $m$}
		% 	% \inputminted[tabsize=2,breaklines,firstline=307,lastline=313,fontsize=\small]{c++}{numberTheory.cpp}
			
		% 	\subsubsection{Test individual de raíz primitiva módulo $m$}
		% 	% \inputminted[tabsize=2,breaklines,firstline=315,lastline=325,fontsize=\small]{c++}{numberTheory.cpp}
			
		% 	\subsubsection{Test individual de raíz $k$-ésima de la unidad módulo $m$}
		% 	% \inputminted[tabsize=2,breaklines,firstline=327,lastline=336,fontsize=\small]{c++}{numberTheory.cpp}
			
		% 	\subsubsection{Encontrar la primera raíz primitiva módulo $m$}
		% 	% \inputminted[tabsize=2,breaklines,firstline=338,lastline=355,fontsize=\small]{c++}{numberTheory.cpp}
			
		% 	\subsubsection{Encontrar la primera raíz $k$-ésima de la unidad módulo $m$}
		% 	% \inputminted[tabsize=2,breaklines,firstline=357,lastline=373,fontsize=\small]{c++}{numberTheory.cpp}
			
		% 	\subsubsection{Logaritmo discreto}
		% 	% \inputminted[tabsize=2,breaklines,firstline=375,lastline=398,fontsize=\small]{c++}{numberTheory.cpp}
			
		% 	\subsubsection{Raíz $k$-ésima discreta}
		% 	% \inputminted[tabsize=2,breaklines,firstline=400,lastline=416,fontsize=\small]{c++}{numberTheory.cpp}
			
	% 	\subsection{Particiones}
	% 		\subsubsection{Función $P$ (particiones de un entero positivo)}
	% 		% \inputminted[tabsize=2,breaklines,firstline=519,lastline=547,fontsize=\small]{c++}{numberTheory.cpp}
			
	% 		\subsubsection{Función $Q$ (particiones de un entero positivo en distintos sumandos)}
	% 		% \inputminted[tabsize=2,breaklines,firstline=549,lastline=596,fontsize=\small]{c++}{numberTheory.cpp}
			
	% 		\subsubsection{Número de factorizaciones ordenadas}
	% 		% \inputminted[tabsize=2,breaklines,firstline=743,lastline=771,fontsize=\small]{c++}{numberTheory.cpp}
			
	% 		\subsubsection{Número de factorizaciones no ordenadas}
	% 		% \inputminted[tabsize=2,breaklines,firstline=773,lastline=799,fontsize=\small]{c++}{numberTheory.cpp}
			
	% \newpage
		% \subsection{Números racionales}
		% 	\subsubsection{Estructura \texttt{fraccion}}
		% 	% \inputminted[tabsize=2,breaklines,firstline=7,lastline=123,fontsize=\small]{c++}{fraccion.cpp}
		
		% \new+page
