\subsection{Linear Algebra}
	\subsubsection{Struct matrix} 
	\pathinputminted[tabsize=2,breaklines,firstline=7,lastline=141,fontsize=\small]{c++}{Algebra_lineal.cpp}
	\pathinputminted[tabsize=2,breaklines,firstline=342,lastline=342,fontsize=\small]{c++}{Algebra_lineal.cpp}
	
	\subsubsection{Transpuesta}
	\pathinputminted[tabsize=2,breaklines,firstline=219,lastline=227,fontsize=\small]{c++}{Algebra_lineal.cpp}
	
	\subsubsection{Traza}
	\pathinputminted[tabsize=2,breaklines,firstline=229,lastline=234,fontsize=\small]{c++}{Algebra_lineal.cpp}
				
	\subsubsection{Gauss System of Linear Equationsn}
	\pathinputminted[tabsize=2,breaklines,firstline=295,lastline=339,fontsize=\small]{c++}{Algebra_lineal.cpp}
	
	\subsubsection{Gauss Determinant}
	\pathinputminted[tabsize=2,breaklines,firstline=262,lastline=293,fontsize=\small]{c++}{Algebra_lineal.cpp}
	
	\subsubsection{Cofactors Matrix}
	\pathinputminted[tabsize=2,breaklines,firstline=194,lastline=200,fontsize=\small]{c++}{Algebra_lineal.cpp}

	\subsubsection{Matriz inversa}
	\pathinputminted[tabsize=2,breaklines,firstline=143,lastline=180,fontsize=\small]{c++}{Algebra_lineal.cpp}
	
	\subsubsection{Adjoint Matrix}
	\pathinputminted[tabsize=2,breaklines,firstline=201,lastline=216,fontsize=\small]{c++}{Algebra_lineal.cpp}
	
	\subsubsection{Recurrencias lineales}
	\pathinputminted[tabsize=2,breaklines,firstline=351,lastline=370,fontsize=\small]{c++}{Algebra_lineal.cpp}
	
	\subsubsection{Kirchhoff Matrix Tree Theorem}
	Count the number of spanning trees in a graph, as the determinant of the Laplacian matrix of the graph. 
	\\
	\textbf{Laplacian Matrix} :
	\\Given a simple graph $G$ with $n$ vertices,
	its Laplacian matrix $L_{n\times n}$ is defined as\\
	\begin{align*}
		L=D-A
	\end{align*}
	The elements of $L$ are given by
	\[ L_{i,j} =
	\begin{cases}
		deg(v_{i})       & \quad \text{if } i == j\\
		-1  & \quad \text{if } i \neq j \text{and } v_{i} \text{ is adjacent to }v_{j}\\
		0 & \quad \text{otherwise}
	\end{cases}
	\]
		define $\tau(G  )$ as number of spanning trees of a grap $G$
	\begin{align*}
		\tau(G) = \det L_{n-1\times n-1}\\
	\end{align*}
	Where  $L_{n-1 \times n-1}$ is a laplacian matrix deleting 
	any row and any column
	\begin{align*}
		\det
		\begin{pmatrix}
			deg(v_{1}) & L_{1,2} & \cdots & L_{1,n-1} \\
		L_{2,1} & deg(v_{2}) & \cdots & L_{2,n-1} \\
		\vdots  & \vdots  & \ddots & \vdots  \\
		L_{n-1,1} & L_{n-1,2} & \cdots & deg(v_{n-1}) 
		\end{pmatrix}
	\end{align*}
	Generalization for a multigraph $K_{n}^{m} \pm G$\\
	define $\tau(K_{n}^{m} \pm G)$ as number of spanning trees of a grap $K_{n}^{m} \pm G$
	\begin{align*}
		\tau(K_{n}^{m} \pm G) = n * (nm)^{n-p-2}\det (B)
	\end{align*}
	where $B = mnI_{p} +\alpha * L(G)$is a $p\times p$ matrix, $\alpha = \pm$  according 
	$(K_{n}^{m} \pm G)$, and $L(G)$ is the Kirchhoff	matrix of G
	\pathinputminted[tabsize=2,breaklines,firstline=422,lastline=438,fontsize=\small]{c++}{Algebra_lineal.cpp}
	% \subsubsection{Linear Recurrence and Berlekamp-Massey Algorithm}
	% \inputminted[tabsize=2,breaklines,firstline=7,lastline=38,fontsize=\small]{c++}{recurrence.cpp}

	% \inputminted[tabsize=2,breaklines,firstline=341,lastline=376,fontsize=\small]{c++}{matrix.cpp}
	
	% \inputminted[tabsize=2,breaklines,firstline=378,lastline=394,fontsize=\small]{c++}{matrix.cpp}
	
	% \subsubsection{Polinomio característico}
	% \subsubsection{Rank of a matrix}
	% \inputminted[tabsize=2,breaklines,firstline=396,lastline=406,fontsize=\small]{c++}{matrix.cpp}
	
	% \subsubsection{Gram-Schmidt}
	% \inputminted[tabsize=2,breaklines,firstline=408,lastline=422,fontsize=\small]{c++}{matrix.cpp}
	