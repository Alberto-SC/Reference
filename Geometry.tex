	% \section{Geometría}
	% 	\subsection{Estructura \texttt{point}}
	% % 	\inputminted[tabsize=2,breaklines,firstline=4,lastline=99,fontsize=\small]{c++}{geometry.cpp}
		
	% 	\subsection{Líneas y segmentos}
	% 		\subsusubbsection{Verificar si un punto pertenece a una línea o segmento}
	% 		\inputminted[tabsize=2,breaklines,firstline=102,lastline=111,fontsize=\small]{c++}{geometry.cpp}
			
	% 		\subsusubbsection{Intersección de líneas}
	% 		\inputminted[tabsize=2,breaklines,firstline=113,lastline=133,fontsize=\small]{c++}{geometry.cpp}
			
	% 		\subsubsection{Intersección línea-segmento}
	% 		\inputminted[tabsize=2,breaklines,firstline=135,lastline=148,fontsize=\small]{c++}{geometry.cpp}
			
	% 		\subsubsection{Intersección de segmentos}
	% 		\inputminted[tabsize=2,breaklines,firstline=150,lastline=167,fontsize=\small]{c++}{geometry.cpp}
			
	% 		\subsubsection{Distancia punto-recta}
	% 		\inputminted[tabsize=2,breaklines,firstline=169,lastline=172,fontsize=\small]{c++}{geometry.cpp}
			
	% 	\subsection{Círculos}
	% 		\subsubsection{Distancia punto-círculo}
	% 		\inputminted[tabsize=2,breaklines,firstline=391,lastline=394,fontsize=\small]{c++}{geometry.cpp}
			
	% 		\subsubsection{Proyección punto exterior a círculo}
	% 		\inputminted[tabsize=2,breaklines,firstline=396,lastline=399,fontsize=\small]{c++}{geometry.cpp}
			
	% 		\subsubsection{Puntos de tangencia de punto exterior}
	% 		\inputminted[tabsize=2,breaklines,firstline=401,lastline=406,fontsize=\small]{c++}{geometry.cpp}
			
	% 		\subsubsection{Intersección línea-círculo}
	% 		\inputminted[tabsize=2,breaklines,firstline=408,lastline=422,fontsize=\small]{c++}{geometry.cpp}
			
	% 		\subsubsection{Centro y radio a través de tres puntos}
	% 		\inputminted[tabsize=2,breaklines,firstline=424,lastline=429,fontsize=\small]{c++}{geometry.cpp}
			
	% 		\subsubsection{Intersección de círculos}
	% 		\inputminted[tabsize=2,breaklines,firstline=431,lastline=448,fontsize=\small]{c++}{geometry.cpp}
			
	% 		\subsubsection{Tangentes}
	% 		\inputminted[tabsize=2,breaklines,firstline=450,lastline=494,fontsize=\small]{c++}{geometry.cpp}
		
	% 	\subsection{Polígonos}
	% 		\subsubsection{Perímetro y área de un polígono}
	% 		\inputminted[tabsize=2,breaklines,firstline=174,lastline=190,fontsize=\small]{c++}{geometry.cpp}
			
	% 		\subsubsection{Envolvente convexa (convex hull) de un polígono}
	% 		\inputminted[tabsize=2,breaklines,firstline=192,lastline=211,fontsize=\small]{c++}{geometry.cpp}
			
	% 		\subsubsection{Verificar si un punto pertenece al perímetro de un polígono}
	% 		\inputminted[tabsize=2,breaklines,firstline=213,lastline=221,fontsize=\small]{c++}{geometry.cpp}
			
	% 		\subsubsection{Verificar si un punto pertenece a un polígono}
	% 		\inputminted[tabsize=2,breaklines,firstline=223,lastline=234,fontsize=\small]{c++}{geometry.cpp}
			
	% 		\subsubsection{Centroide de un polígono}
	% 		\inputminted[tabsize=2,breaklines,firstline=264,lastline=274,fontsize=\small]{c++}{geometry.cpp}
			
	% 		\subsubsection{Pares de puntos antipodales}
	% 		\inputminted[tabsize=2,breaklines,firstline=341,lastline=352,fontsize=\small]{c++}{geometry.cpp}
			
	% 		\subsubsection{Diámetro y ancho}
	% 		\inputminted[tabsize=2,breaklines,firstline=354,lastline=368,fontsize=\small]{c++}{geometry.cpp}
			
	% 		\subsubsection{Smallest enclosing rectangle}
	% 		\inputminted[tabsize=2,breaklines,firstline=370,lastline=389,fontsize=\small]{c++}{geometry.cpp}
		
	% 	\subsection{Par de puntos más cercanos}
	% 	\inputminted[tabsize=2,breaklines,firstline=236,lastline=262,fontsize=\small]{c++}{geometry.cpp}
		
	% 	\subsection{Vantage Point Tree (puntos más cercanos a cada punto)}
	% 	\inputminted[tabsize=2,breaklines,firstline=276,lastline=339,fontsize=\small]{c++}{geometry.cpp}
		
	% 	\subsection{Suma Minkowski}
	% 	\inputminted[tabsize=2,breaklines,firstline=496,lastline=517,fontsize=\small]{c++}{geometry.cpp}
		